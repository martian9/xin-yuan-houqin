% !Mode:: "TeX:UTF-8"

\begin{Cabstract}{众核}{动态热管理}{模型预测控制}{任务迁移}{动态电压频率调整}

随着众核处理器的发展,功耗密度还在不断上升,并且使负载不均衡问题更加严重,形成了严重的芯片高温和局部高温点问题。
高温热问题已经成为阻碍处理器发展的重要因素。为应对芯片热问题,本文做了深入研究。
研究主要集中在动态温度管理方法。针对最近的动态温度管理方法做了详细研究分析。
对动态温度管理常用的操作技术,动态电压频率调整技术和任务迁移技术,做了深入的分析。
对动态温度管理操作需要的引导控制技术也进行了分析,尤其是模型预测控制方法。
HotSpot热模型是一种简洁准确的热建模方法,本文提出的方法中用其来对芯片做热模型,并结合模型预测控制方法进行温度计算。
引入模型预测控制方法之后就可以将任务迁移策略的决定化作任务分配问题,用二部图匹配方法来解决该问题。

因为二部图匹配算法复杂度较高,随着核数的增长算法计算时间显著增加,所以在众核处理器高性能运行时进行热管理是一个巨大的难题。
本文提出分层的动态温度管理的解决方法来克服这个问题。
新的动态温度管理方法依然采用传统的动态电压频率调整技术和任务迁移,采用模型预测控制方法做理论依据来实现平滑的任务控制,减少牺牲计算性能。
为了扩展到众核系统,分层控制方法设计为两层。
在低层,核在空间上被分割成块,块内用现有的功耗分布和用模型预测控制方法优化得到的功耗分布进行匹配。
在高层,对低层没有匹配上的功耗进行全局任务迁移。如果高层的功耗数量很大,本文采用改进的迭代最小割算法来辅助实现任务迁移的策略。
任务迁移操作完成之后动态电压频率调整技术用来调节最终都没有匹配上的功耗。
最后仿真实现该算法,并将其与其他算法进行比较,着重比较算法运行时间,处理器性能和芯片热平衡性。
结果显示,新的方法比现有的其他方法有很大优势,而且在众核处理器中只有很少的性能损耗。
\end{Cabstract}

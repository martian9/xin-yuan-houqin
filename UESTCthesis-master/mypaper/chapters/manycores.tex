% !Mode:: "TeX:UTF-8"

\chapter{绪论}
根据摩尔定律芯片上的晶体管数量每18个月翻一番。随着这些数量空前庞大的晶体管集成在一个芯片上,当前的多核技术很快就会发展成上百核上千核的时代\cite{borkar:DAC'07}。
已经有几十核的芯片投入生产,包括Tilera的64核处理器和intel的至强融核处理器。2012年Intel就发布了名为 Knights Corner 的至强融核协处理器,中国天河二号超级计算机就装了48000个Knights Corner芯片。
Knights Corner芯片上最多可容 61个内核。
intel下一代众核处理器 Knights Landing 可容下更多的核心。
多核和众核技术带来了极大的性能提升,但是我们不得不面对随之而来的功耗和热问题。
因为我们持续减小芯片大小和要求高功耗下的性能,增长的芯片复杂性和功耗密度提高了芯片的峰值温度,也使温度梯度更加不均衡。
上升的峰值温度缩短芯片寿命,降低芯片性能,影响可靠性也增加散热成本\cite{skadron:TACO'04}。上升的温度和静态功耗之间有正反馈的关系,有可能造成热失控。
在多核或者众核系统中,不同的应用负载或许引起核之间功耗和温度的不平衡。温度在时间和空间上的变化产生的芯片局部温度最大值叫做高温点\cite{Donald:ISCA'06}。
过多的空间上的温度差也就是热梯度增加时钟抖动降低性能和可靠性。
上升的温度需要更多的散热能力去冷却处理器,一个典型的散热风扇会消耗高达服务器 $51\%$的功耗\cite{lefurgy2003energy}\cite{ayoub2010gentlecool}。
\section{温度不被管理的不利影响}\label{sec:adverse}
\subsection{温度对系统影响}\label{sec:reliability}
高功耗的一个最明显的结果就是上升的芯片温度,高温对芯片最严重的的后果就是损害芯片的可靠性。
下面是温度相关的半导体器件失效机理\cite{jedec2003failure}:


%我们已经进入众核处理器时代\cite{MaWang:APCCAS'14}.
%2012年,英特尔正式将第一代集成众核产品,代号为Knights Corner的协处理器推向市场。
%这一代产品使用P54C核心构建,片上集成61个CPU核心,实现了GPU级别的浮点运算速度。
%英特尔至强披™协处理器X100系列(代号为“Knights Corner”)是第一代英特尔®集成众核(英特尔®MIC)架构,
%它结合了许多英特尔的CPU内核在单一芯片上的。
%该生产线是针对高度并行的工作负载在各种领域,如计算物理,化学,生物和金融服务。
%Knights Corner片上集成了高达61核心。
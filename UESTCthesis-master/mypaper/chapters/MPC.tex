% !Mode:: "TeX:UTF-8"

\chapter{基于动态温度管理的模型预测控制}

在这一章我们将介绍基于动态温度管理方法的模型预测控制方法。
结合mpc的热模型在\ref{sec:therm_model}节中介绍。2.2节介绍怎么用mpc计算期望的用于引导动态温度管理方法功耗。最后2.3节说明怎么样用mpc计算得到的功耗来引导任务迁移和DVFS。
\section{微处理器热模型}\label{sec:therm_model}
热系统和电路系统是相似的,我们可以用热阻,热容,和等效的热电流电压源来建立微处理器的热模型。
类似于电路系统,$l$ 核的微处理器热模型可以被表达为常微分方程,
\begin{equation}\label{eq:therm_model_cont} 
\begin{split}
G T(t) + C \dot{T}(t) &= B_c P(t),\\
Y(t) &= L T(t),
\end{split}
\end{equation}
其中,$T(t) \in \mathbb{R}^n$ 是表示处理器 $n$ 块温度的向量,包括 $l$个核($l < n$), 边界节点和封装部分的节点;
$G \in \mathbb{R}^{n\times n}$ 包含热阻信息; 
$C \in \mathbb{R}^{n \times n}$ 包含热容信息;
$B_c \in \mathbb{R}^{n \times l}$ 包括功率输入的拓扑信息;
$P(t) \in \mathbb{R}^{l}$ 是 $l$ 个核在时刻 $t$ 的功耗向量,这就是模型的输入;
$Y(t)$ 是 $l$ 个核的温度信息向量,这就是模型的输出;
$L \in \mathbb{R}^{l \times n}$ 是输出选择矩阵,从 $T(t)$ 中选择 $l$ 个核的温度。

为了分析热系统,用欧拉方法或者其他数值积分方法将连续常微分方程 \eqref{eq:therm_model_cont} 离散化为下面的差分方程
\begin{equation}\label{eq:therm_model_disc}
\begin{split}
T(k+1) &= A T(k)+B_d P(k),\\
Y(k) &= L T(k),
\end{split}
\end{equation}
其中,变量 $T(k)$ 、 $P(k)$ 和 $Y(k)$ 是公式 \eqref{eq:therm_model_cont}中
$T(t)$ 、 $P(t)$ 和 $Y(t)$ 的离散形式,  $A$ 和
$B_d$ 是由 $G$、 $C$、 和 $B_c$ 根据离散\eqref{eq:therm_model_cont}的特定的数值积分方法得到的。

对于一般用途, \eqref{eq:therm_model_disc} 中的热模型是用于用芯片上的各单元功耗(即输入$P(k)$)来计算芯片上核的温度(即输出的 $Y(k)$)。

\section{用模型预测控制方法计算期望的功耗}\label{sec:mpc}
\ref{sec:therm_model}节中已经说明,用热模型 \eqref{eq:therm_model_disc}, 可以由给定的功耗输入 $P(k)$ 来计算
芯片上的核的温度 $Y(k)$, 这足以进行热估计和仿真。
对于动态温度管理问题,由给定的功耗来计算期望的功耗也是很重要的,因为动态温度管理方法需要操作功耗方面来管理温度。
有的时候为了简化,可以利用静态热模型由给定的温度信息来计算功耗,静态热模型可以由模型 \eqref{eq:therm_model_cont} 去掉热容项来得到。
然而,基于静态热模型的动态温度管理方法会忽略掉当前热状态,但是当前的热状态在做管理决策的时候非常重要。
这个方法也假设温度和功耗大致温度,这样会影响动态温度管理效用。
为了减轻这个问题,\eqref{eq:therm_model_cont}中的瞬态热模型(或者\eqref{eq:therm_model_disc}中的离散形式)






























% !Mode:: "TeX:UTF-8"

\chapter{基于动态温度管理的模型预测控制}

在这一章我们将介绍基于动态温度管理方法的模型预测控制方法。
结合mpc的热模型在\ref{sec:therm_model}节中介绍。2.2节介绍怎么用mpc计算期望的用于引导动态温度管理方法功耗。最后2.3节说明怎么样用mpc计算得到的功耗来引导任务迁移和DVFS。
\section{微处理器热模型}\label{sec:therm_model}
热系统和电路系统是相似的,我们可以用热阻,热容,和等效的热电流电压源来建立微处理器的热模型。
类似于电路系统,$l$ 核的微处理器热模型可以被表达为常微分方程,
\begin{equation}\label{eq:therm_model_cont} 
\begin{split}
G T(t) + C \dot{T}(t) &= B_c P(t),\\
Y(t) &= L T(t),
\end{split}
\end{equation}
其中,$T(t) \in \mathbb{R}^n$ 是表示处理器 $n$ 块温度的向量,包括 $l$个核($l < n$), 边界节点和封装部分的节点;
$G \in \mathbb{R}^{n\times n}$ 包含热阻信息; 
$C \in \mathbb{R}^{n \times n}$ 包含热容信息;
$B_c \in \mathbb{R}^{n \times l}$ 包括功率输入的拓扑信息;
$P(t) \in \mathbb{R}^{l}$ 是 $l$ 个核在时刻 $t$ 的功耗向量,这就是模型的输入;
$Y(t)$ 是 $l$ 个核的温度信息向量,这就是模型的输出;
$L \in \mathbb{R}^{l \times n}$ 是输出选择矩阵,从 $T(t)$ 中选择 $l$ 个核的温度。

为了分析热系统,用欧拉方法或者其他数值积分方法将连续常微分方程 \eqref{eq:therm_model_cont} 离散化为下面的差分方程
\begin{equation}\label{eq:therm_model_disc}
\begin{split}
T(k+1) &= A T(k)+B_d P(k),\\
Y(k) &= L T(k),
\end{split}
\end{equation}
其中,变量 $T(k)$ 、 $P(k)$ 和 $Y(k)$ 是公式 \eqref{eq:therm_model_cont}中
$T(t)$ 、 $P(t)$ 和 $Y(t)$ 的离散形式,  $A$ 和
$B_d$ 是由 $G$、 $C$、 和 $B_c$ 根据离散\eqref{eq:therm_model_cont}的特定的数值积分方法得到的。

对于一般用途, \eqref{eq:therm_model_disc} 中的热模型是用于用芯片上的各单元功耗(即输入$P(k)$)来计算芯片上核的温度(即输出的 $Y(k)$)。

\section{用模型预测控制方法计算期望的功耗}\label{sec:mpc}
\ref{sec:therm_model}节中已经说明,用热模型 \eqref{eq:therm_model_disc}, 可以由给定的功耗输入 $P(k)$ 来计算
芯片上的核的温度 $Y(k)$, 这足以进行热估计和仿真。
对于动态温度管理问题,由给定的功耗来计算期望的功耗也是很重要的,因为动态温度管理方法需要操作功耗方面来管理温度。
有的时候为了简化,可以利用静态热模型由给定的温度信息来计算功耗,静态热模型可以由模型 \eqref{eq:therm_model_cont} 去掉热容项来得到。
然而,基于静态热模型的动态温度管理方法会忽略掉当前热状态,但是当前的热状态在做管理决策的时候非常重要。
这个方法也假设温度和功耗大致温度,这样会影响动态温度管理效用。
为了减轻这个问题,一些反馈控制方案或者优化设计用\eqref{eq:therm_model_cont}中的瞬态热模型(或者\eqref{eq:therm_model_disc}中的离散形式)
在动态温度管理决策时进行更好的功耗计算。
尽管这个方法考虑了当前温度状态而且处理了热与功率的影响,但是这种方法不能得到一个平滑的温度控制。
主要是因为这种方法缺乏未来预测能力,而且只能为温度控制获得当前步的优化功耗。
在这篇文章中,我们用了基于模型预测控制功率计算方法,这个方法将\eqref{eq:therm_model_disc} 中的瞬态热模型扩展成预测形式,它具备为平滑精确热管理计算未来期望功耗的能力。
模型预测控制方法利用\eqref{eq:therm_model_disc} 中的系统模型可以计算得到输入的调整需求,这样就能满足设计者定义的输出。
为了最大化处理器性能,处理器每个核允许的最高温度称作顶温度$Y_{max}$, $Y_{max}$ 通常当做设计者定义的输出来被趋近。
顶温度可以根据现实中的不同应用来被调整,它可以稍微低于处理器允许的最高温度以保证绝对的安全。

首先,我们定义状态和温度变量的差
\begin{equation}
\begin{split}
\Delta T(k) &= T(k) - T(k-1), \\
\Delta P(k) &= P(k) - P(k-1).
\end{split}
\end{equation}
取\eqref{eq:therm_model_disc} 的相邻两步的差,这里有
\begin{equation}\label{eq:diff_model}
\begin{split}
\Delta T(k+1) &= A \Delta T(k) + B_d \Delta P(k),\\
Y(k+1)-Y(k) &= L A \Delta T(k) + L B_d \Delta P(k).
\end{split}
\end{equation}
引入一个新的变量
\begin{equation*}
\hat{T}(k) =
\left[
\begin{array}{c}
\Delta T(k)\\
Y(k)
\end{array}
\right],\\
\end{equation*}
将 \eqref{eq:diff_model} 重写成下面改进的模型
\begin{equation}\label{eq:aug_model}
\begin{split}
\hat{T}(k+1) &= \hat{A}\hat{T}(k) + \hat{B}\Delta P(k),\\
Y(k) &= \hat{L}\hat{T}(k),
\end{split}
\end{equation}
其中
\begin{align*}
\hat{A} &= 
\left[
\begin{array}{cc}
A & 0_m\\
L A & I
\end{array}
\right], &
\hat{B} &= 
\left[
\begin{array}{c}
B_d\\
L B_d
\end{array}
\right],\\
\hat{L} &= 
\left[
\begin{array}{cc}
0_m & I
\end{array}
\right], &
\hat{T}(k) &= 
\left[
\begin{array}{c}
\Delta T(k)\\
Y(k)
\end{array}
\right],
\end{align*}
$0_m$ 是一个合适维度的全零矩阵。


到这里我们已经从\eqref{eq:aug_model}中得到了输入功耗差和输出核的温度之间的关系。
下面,需要确定输入功耗的差来满足核期望的顶温度。假设核未来几个时间步长的顶温度已经给出,写成下面向量的形式
\begin{equation*}
Y_{ceil} = [Y_{max}^T, Y_{max}^T, ..., Y_{max}^T]^T \in \mathbb{R}^{lN_p \times 1}.
\end{equation*}
在这个向量中,$Y_{max} \in \mathbb{R}^{l \times 1}$ 包含每一个核的顶温度。
这里我们假设顶温度不变,这个也符合实际情况,并不是新方法的限制。
$N_p$ 表示从当前到未来$N_p$ 步的时间帧,称作预测域。
为了使核的温度在时间域内趋近于顶温度,将来的控制轨迹(并不知道,需要计算)表示为(当前时刻为 $k$ )
\begin{equation*}
\Delta P_k = [\Delta P(k), \Delta P(k+1), \ldots, \Delta P(k+N_c-1)]^T,
\end{equation*}
其中,$N_c$ 称作控制域。核的预测温度定义为
\begin{equation*}
Y_k = [Y(k+1|k)^T, Y(k+2|k)^T, \ldots, Y(k+N_p|k)^T]^T,
\end{equation*}
其中, $Y(k+j|k)$ 利用当前时刻$k$的信息预测出来的核在时刻 $k+j$ 的温度。
如果$\Delta P_k$已知,$Y_k$ 就可以用下面的公式计算出来
\begin{equation}\label{eq:pred_eq}
Y_k = V\hat{T}(k) + \Phi \Delta P_k,
\end{equation}
其中 
\begin{equation*}
V = 
\left[
\begin{array}{c}
\hat{L}\hat{A}\\
\hat{L}\hat{A}^2\\
\vdots\\
\hat{L}\hat{A}^{N_p}
\end{array}
\right],
\Phi = 
\left[
\begin{array}{ccccc}
\hat{L} \hat{B} & 0  & 0 & \cdots & 0\\
\hat{L} \hat{A} \hat{B} & \hat{L} \hat{B} & 0 & \cdots & 0 \\
\hat{L} \hat{A}^2 \hat{B} & \hat{L} \hat{A} \hat{B} & \hat{L} \hat{B} & \cdots & 0\\
\vdots &  \vdots & \vdots & \ddots & \vdots \\
\hat{L} \hat{A}^{N_p-1} \hat{B} & \hat{L} \hat{A}^{N_p-2} \hat{B} &
\hat{L} \hat{A}^{N_p-3} \hat{B} & \cdots & \hat{L} \hat{A}^{N_p-N_c} \hat{B}
\end{array}
\right].
\end{equation*}
接下来,我们想计算功耗,使利用这个功耗计算出的核的温度$Y_k$  和设计者定义的期望的顶温度$Y_{ceil}$ 之间的差最小。
我们首先将这个差的测量值表示为$(Y_{ceil}-Y_k)^T(Y_{ceil}-Y_k)$ , 最优的功耗分布是使$Y_k=Y_{ceil}$ 的功耗。
此外,对于实际的考虑,我们优先选择功耗分布不进行急剧变化。
所以额外的调整项 $\Delta P_k^TR\Delta P_k$ 也加到$(Y_{ceil}-Y_k)^T(Y_{ceil}-Y_k)$ 上,这就形成
\begin{equation}\label{eq:cost_fun}
F = (Y_{ceil}-Y_k)^T(Y_{ceil}-Y_k)+\Delta P_k^TR\Delta P_k
\end{equation}
作为变量$\Delta P_k$ 的最终函数,我们的目标就是使这个函数最小化。
$R=rI_{N_c \times N_c}$ 是一个调整矩阵,$r$ 是调整参数,这决定该函数两项之间的权重。
对于不同的核数通过实验可以的到合适的值。
这里要注意,$Y_k$ 也是未知变量$\Delta P_k$ 的函数。

下一步,对式\eqref{eq:cost_fun}求一阶导数,使它等于零,就可以得到优化的最小值。
$\Delta P_k$的解是
\begin{equation}\label{eq:opt_delta_p}
\Delta P_k = (\Phi^T \Phi + R)^{-1}\Phi^T(Y_{ceil}-V\hat{T}(k)).
\end{equation}

在每个模型预测控制MPC时刻 $k$,我们只需要从\eqref{eq:opt_delta_p}计算得到的控制信号$\Delta P(k)$, 
然后更新功耗分布
\begin{equation}\label{eq:power_update}
\bar{P}(k) \gets P(k) + \Delta P(k),
\end{equation}
其中$\bar{P}(k)$  是更新后的功耗分布。
结果就是,更新后功耗输入使核的温度 $Y(k)$ 趋近于期望顶温度。
换句话说,更新后的功耗是在没有温度要求冲突下能达到的最高温度。

\section{基于期望功耗的任务迁移和动态电压频率调整}\label{sec:dtm_mpc}

































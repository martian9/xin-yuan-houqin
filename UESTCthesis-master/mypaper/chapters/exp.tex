% !Mode:: "TeX:UTF-8"

\chapter{实验结果}\label{sec:exp}

这个实验是在一个具有两个8核16线程CPU的linux服务器上进行的,每个CPU主频为2.90GHZ,服务器内存64GB。
新的分层动态热管理方法主要用MATLAB实现。
我们分别构建了四个不同的处理器核配置的众核处理器,从$100$核($10 \times 10$)到$625$核($25 \times 25$)。
这些处理器的热模型由HotSpot生成。环境温度设定为$20^{\circ}$C。
在这个实验中众核处理器由完全一致的Alpha 21264核组成。所有芯片的大小都是$10mm \times 10mm \times 0.15mm$。
这里我们假设众核处理器中任务运行时相互之间没有通讯也不需要同步。
任务的功耗由Wattch通过运行SPEC基准程序生成,初始任务分配为,一个任务随机指定给一个核运行。
接下来的任务分配和调度有动态温度管理方法确定。
我们有9个SPEC基准程序的实时功耗信息。
对于不同的处理器的功耗信息,我们重复这9个功耗信息得到处理器需要的100个实时功耗信息,256个实时功耗信息等等。

因为核的大小会随核的数量增长变化,这会超过所谓的“功耗墙”或者“利用率墙”导致不现实的功耗密度,产生极高的温度。
为解决这个问题,提出了很多解决方法。一个解决方案是灰硅,放缩每个核的功耗。另一个更广泛的方法是完全关掉一些核。
在我们的研究中,我们采用灰硅技术,放缩功耗的大小以保证所有处理器都有相似的功耗密度,这样温度分布就类似于现在的多核芯片。
这个可以通过以一定的比例调整操作频率和电压来实现。
在我们的这个研究中,我们并不考虑完全关掉一部分核的策略,这会是我们将来的研究方向。





































% !Mode:: "TeX:UTF-8"

\chapter{总结}

近年来,处理器技术持续发展,已经由多核处理器发展到众核领域。随着众核处理器带来的性能提升,也带来了一定的负面影响。
尤其是更好密度增加,出现了热问题。不仅有高温问题,众核处理工作负载不均衡,导致处理器温度梯度加大,极可能出现局部高温点。
高温和高的温度梯度都对处理器有极大的影响,会导致处理器性能和可靠性下降,影响芯片寿命,冷却成本也急剧上升。
针对这一问题,本文做了动态温度管理方面的研究,提出了一种针对高性能众核系统的新的分层动态温度管理方法。
动态温度管理方法通过调整任务的执行改变功耗特性,平衡芯片负载,降低峰值温度和局部高温点数量,能有效改善热问题。

本文首先分析了芯片温度不受管理对芯片的各方面负面影响,温度过高对芯片可靠性,芯片性能到来极大损害。
过高的温度也会加剧静态功耗增加,静态功耗的增加反过来又会加剧温度上升。现在的冷却系统对于处理现有的高温问题已经相当困难,需要极高的冷却成本。
对于动态温度管理方法,虽然是对任务或者处理器进行调整来修正功耗特性,但是这不仅仅与功耗相关,与芯片封装和散热参数等等都有很大的关联。
现今动态温度管理方法最常用的操作 动态电压频率调整和任务迁移技术,在修正功耗或功耗分布方面很有作用。对这些方法如何能调整修正温度也进行了详细说明。
仅仅有动态温度管理操作是不能够有效控制芯片温度的,这些操作需要一个良好的引导控制。所以引导控制需要的温度预测方法和模型预测控制方法也在这里进行了介绍。
本文提出的新的分层动态温度管理方法就采用了模型预测控制方法,并结合了动态电压频率调整和任务迁移技术。

对于众核处理器进行模型预测控制,首先需要对芯片进行热建模。 HotSpot 是一种简洁精确的热建模方法。
热系统与电路系统有相似的对偶关系,对热传导理论进行说明。并且为方便理解处理器热模型,对 HotSpot 热建模方法进行了介绍,
并讨论了热模型在动态温度管理方法中的作用。

本文提出了针对高性能多核众核处理器的动态热管理算法。对其进行了详细介绍,新方法基于模型预测控制,
结合了任务迁移和动态电压频率调整技术,能有效降低性能损耗,提高芯片可靠性。
做任务迁移决策时用到的复杂度较高的匈牙利算法做二部图匹配,直接扩展到众核系统需要的计算开销很大。
为了扩展到众核系统,在两个层次上做基于二部图匹配的任务迁移决策:
块内的低层,块间的高层。
改进的最小割算法用于辅助高层的任务迁移决策过程。
在实验中建模实现该算法,并在不同核数的众核处理器上验证测试,并与其它动态温度管理方法进行比较。
新方法能够把芯片温度控制在安全范围内,能有效解决众核芯片热问题,并且能使处理器得到更高的计算性能,比现有的方法具有很大优势。

最后本文方法也有一定的局限性,在仿真实现中做了一些理想化的假设,比如动态电压频率调整时,可以将频率和电压连续地调整到任意水平。
实际可以调整频率电源电压的处理器一般只设定几个离散的频率水平。所以这方面需要在未来的工作中改进。
还有文中提到的 DVFS 要用到的 DC-DC 转换器,为减少转换器的开销,可以考虑几个核共用一个转换器,这样就要改进 DVFS 策略。
同样的这部分改进也需要在未来工作中做进一步研究。






























